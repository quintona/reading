\documentclass[11pt]{article}
\usepackage{listings}
\usepackage{hyperref}
\usepackage{tikz-cd}
\usepackage{dcpic}
\usepackage{tikz}
\usetikzlibrary{matrix}
\usepackage{tkz-graph}
\usetikzlibrary{arrows}

% "define" Scala
\lstdefinelanguage{Scala}{
  morekeywords={abstract,case,catch,class,def,%
    do,else,extends,false,final,finally,%
    for,if,implicit,import,match,mixin,%
    new,null,object,override,package,%
    private,protected,requires,return,sealed,%
    super,this,throw,trait,true,try,%
    type,val,var,while,with,yield},
  otherkeywords={=>,<-,<\%,<:,>:,\#,@},
  sensitive=true,
  morecomment=[l]{//},
  morecomment=[n]{/*}{*/},
  morestring=[b]",
  morestring=[b]',
  morestring=[b]"""
}

\begin{document}
\title{Reading Notes}
\author{Quinton Anderson}
\date{\today}
\abstract{These are my notes, that I make as I read articles}
\maketitle

\section{Typesetting Sandbox}

and here: $ A \circ B $ 
Here is a list of useful links:
\begin{itemize}

  \item \href{http://docs.mathjax.org/en/latest/tex.html}{Tex Info}
  \item \href{http://www.jmilne.org/not/Mamscd.pdf}{amscd Tex extension guide}
  \item \href{https://latex.codecogs.com/}{Play with Latex Math}
  \item \href{http://www.sciweavers.org/free-online-latex-equation-editor}{Free Online Editor}

\end{itemize}


\section{Scala Notes}

The scala Option.fold only folds in a value if its empty

\lstset{language=Scala} 
\begin{lstlisting}  % Start your code-block

val s = List(1,2,3)
s.map(_ + 1)

\end{lstlisting}

\section{Bibtex}

An excellent reference is simply on \href{http://en.wikibooks.org/wiki/LaTeX/Bibliography_Management}{wikipedia}

Here are some templates for the .bib file: 

\begin{lstlisting}	
@article{Xarticle,
    author    = "",
    title     = "",
    journal   = "",
    %volume   = "",
    %number   = "",
    %pages    = "",
    year      = "XXXX",
    %month    = "",
    %note     = "",
}
\end{lstlisting}

And here is an example of citing an entry from the .bib file:  \cite{greenwade93}

\bibliographystyle{plain}
\bibliography{test.bib}
\end{document}